\section{Introduction}
\subsection{Contexte}
Souvent, il arrive qu'on ait besoin de l'informatique pour résoudre des problèmes. Ici, c'est un de ces cas là : un fabricant doit déterminer la façon de produire et livrer au client les produits commandés la plus efficace. La dimension du problème fait que la résolution à la main est impossible. Il nous incombe donc de développer une façon de traiter ce problème, et d'en sortir la, sinon une, bonne solution pour le fabricant. La structure du problème nous appelle immédiatement à l'utilisation de divers algorithmes et de recherches opérationnelles, c'est donc dans cette voie que le sujet se développe. 

\subsection{Informations fournies}
Plusieurs contraintes nous ont été imposées. Celles-ci peuvent être retrouvés sous leurs formulation mathématique dans le sujet du challenge, donc on ne les ré-écrira pas ici. Ce qui nous a aussi été donné  est une structure du problème en Java, dans lequel notre projet devait s'inscrire, et un script de lancement.
Nous avons aussi reçu plusieurs instances de problème à résoudre, de tailles variés.

\subsection{Approche}
Avec ceci, nous avons pu prendre une première approche du problème. L'aspect concret des classes Java a rendu la compréhension beaucoup plus simple que la simple formulation abstraite des conditions mathématiques, et nous avons pu assez rapidement commencer à explorer les possibilités de résolution.