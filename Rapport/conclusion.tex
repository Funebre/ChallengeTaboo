\section{Conclusion}
Voilà en somme notre approche du challenge qui nous a été donné. Ce qui est intéressant à remarquer avec notre solution, c'est que nous obtenons de très bons résultats en très peu de temps pour les problèmes de tailles petites et intermédiaires, mais que, pour les problèmes à plus grande dimension, nous trouvons rarement de bons résultats, et l'augmentation du temps de recherche n'y change que peu de chose.\\

\vspace{1em}

Une chose intéressante à faire serait peut-être de revoir la fonction de voisinage pour en définir un plus adapté à l'échelle des grandes instances. Là, le tâtonnement pourrait peut-être plus efficace et chercher plus loin autour de la meilleure solution trouvée par l'algorithme évolutionniste.\\

\vspace{1em}

Quoiqu'il en soit, vous pouvez trouver quelques résultats ci-après.